\newcommand{\R}{\mathbb{R}}
\newcommand{\N}{\mathbb{N}}
\newcommand{\Nb}[1]{\ensuremath{\N_{#1}}}
\newcommand{\Nt}{\Nb{2}}
\newcommand{\range}[1]{\ensuremath{\left[\left[ #1 \right]\right]}}
\newcommand{\Square}[1]{\ensuremath{sq({#1})}}
\newcommand{\tuple}[1]{\ensuremath{\left\langle #1 \right\rangle}}
\newcommand{\Set}[1]{\ensuremath{\left\{ #1 \right\}}}
\newcommand{\SetOf}[2]{\ensuremath{\Set{ #1 : #2}}}
\newcommand{\tupleOf}[2]{\ensuremath{\tuple{ #1 \ : \ #2}}}
\newcommand{\incl}[2]{#1 \triangleright #2}
\newcommand{\great}{Great\xspace}

\newcommand{\dref}[1]{Definition \ref{#1}\xspace}
\newcommand{\tref}[1]{Theorem \ref{#1}\xspace}

\newcommand{\op}{\ensuremath{\omega}}
\renewcommand{\b}[1]{\ensuremath{b\left( #1 \right)}}
\newcommand{\s}[1]{\ensuremath{s\left( #1 \right)}}
\newcommand{\price}[1]{\ensuremath{\rho\left( #1 \right)}}
\newcommand{\OS}{\ensuremath{\Omega_{n}}}
\newcommand{\pricep}[1]{\ensuremath{p_{\s{#1}} - p_{\b{#1}}}}

\newcommand{\sopt}{\ensuremath{S^{*}}}
\newcommand{\sop}{\ensuremath{\op^{*}}}

\chapter{Share Market}

\section{Basic Definitions}

\begin{defn}[Natural]
    Given $v \in \N$, we define:
    \begin{equation}
        \Nb{v} = \SetOf{n \in \N}{n \geqslant v}
    \end{equation}
\end{defn}

\begin{defn}[Range]
    \label{def:range}
    Given $n \in \Nt$, we define the \textbf{Range of $n$} as:
    \begin{equation}
        \range{n} = \Set{0, \dots, n - 1} = \SetOf{i \in \N}{i < n}
    \end{equation}
\end{defn}

\begin{defn}[Square]
    \label{def:square}
    Given $n \in \Nt$, we define the \textbf{Square of $n$} as:
    \begin{equation}
        \Square{n} = \range{n} \times \range{n}
    \end{equation}
\end{defn}

\section{Problem Definitions}

You are given an array in which the $i$th element is the price of a given stock on the day $i$. You are permitted to complete at most 1 transaction (i.e. buy once and sell once). What is the maximum profit you can gain?

Notice that you cannot sell a stock before buying it.

\subsection{Definitions}

\begin{defn}[Price Tuple]
    \label{def:price}
    Given $n \in \Nt$, a \textbf{Price Tuple} $p$ is a positive real tuple with $n$ elements:
    \begin{equation}
        p = \langle p_{0}, \dots, p_{n-1} \rangle \in \R^n_+
    \end{equation}
\end{defn}

\begin{defn}[Operation Pair]
    \label{def:op-pair}
    Given $n \in \Nt$, an \textbf{Operation Pair} $\op$ is a pair:
    \begin{equation}
        \op = \tuple{\b{\op}, \s{\op}} \in \Square{n}
        \quad ,
        b < s
    \end{equation}
    when the context is clear enough, we will simply write $\op = \tuple{\b{\op}, \s{\op}} = \tuple{b, s}$.
\end{defn}

\begin{defn}[Ascending Operation Pair]
    Given $n \in \Nt$ and a Price Tuple $p$, an Operation Pair $\op = \tuple{b, s}$ is said to be \textbf{Ascending} when:
    \begin{equation}
        p_{b} \leqslant p_{b+1} \leqslant \dots \leqslant p_{s-1} \leqslant p_{s}
    \end{equation}
\end{defn}

\begin{defn}[Maximal Operation Pair]
    \label{def:max-op-pair}
    Given $n \in \Nt$ and a Price Tuple $p$, we say that an Ascending Operation Pair $\op = \tuple{b, s}$ is a \textbf{Maximal Operation Pair}, or simply that $\op$ is \textbf{Maximal}, when it is Ascending and it satisfies the conditions:
    \begin{align}
        p_s - p_b \geqslant p_{s+1} - p_b \quad \mbox{, if } s+1 < n\\
        p_s - p_b \geqslant p_s - p_{b-1} \quad \mbox{, if } b-1 > 0
    \end{align}
\end{defn}

\begin{theorem}
    \label{theorem:max-op-pair}
    Given $n \in \Nt$ and a Price Tuple $p$, an Operation Pair $\op = \tuple{b, s}$ is Maximal if and only if the following conditions are satisfied:
    \begin{align}
        p_s \geqslant p_{s+1} \quad \mbox{, if } s+1 < n\\
        p_b \leqslant p_{b-1} \quad \mbox{, if } b-1 > 0
    \end{align}
\end{theorem}

\begin{proof}
    It comes directly from the inequalities of the \dref{def:max-op-pair}.
\end{proof}

\begin{defn}[Independent Operation Pairs]
    Given two Operation Pairs $\op, \op'$, we say that $\op, \op'$ are \textbf{Independent} when:
    \begin{equation}
        \s{\op} < \b{\op'} \lor \s{\op'} < \b{\op}
    \end{equation}
    moreover, given a set of Operation Pairs $S = \Set{\op_0, \dots, \op_{m-1}}$, we say that $S$ is independent when it satisfies:
    \begin{equation}
        \forall \op \forall \op' \left(
            \op \in S \land \op' \in S
            \rightarrow
            \op, \op' \mbox{ are independent}
        \right)
    \end{equation}
    i.e. all Operation Pairs $\op, \op' \in S$ are pairwise Independent.
\end{defn}

\begin{defn}[Operation Set]
    Given $n \in \Nt$, an \textbf{Operation Set} $S \subseteq \Square{n}$ is a set which satisfies:
    \begin{enumerate}
        \item $\forall \omega \left( \omega \in S \rightarrow \omega \mbox{ is an Operation Pair} \right)$
        \item $S$ is independent
    \end{enumerate}
    Moreover, we denote the set of all possible Operations Set by:
    \begin{equation}
        \OS = \SetOf{S \subseteq \Square{n}}{S \mbox{ is an Operation Set}}
    \end{equation}
\end{defn}

\begin{defn}
    Given an $n \in \Nt$, a Price Tuple $p$, and an Operation Set $S \in \OS$, the price $\price{S}$ of $S$ is:
    \begin{equation}
        \price{S} = \displaystyle\sum\limits_{\op \in S} p_{\s{\op}} - p_{\b{\op}}
    \end{equation}
\end{defn}

\begin{lemma}
    \label{lemma:opt-price-I}
    Given an $n \in \Nt$ and a Price Tuple $p$, if a Operation Set $\sopt \in \OS$ is has optimal price, then all Operation Pairs of $\sopt$ are Maximal, i.e.:
    \begin{align}
        \price{\sopt} = \max\limits_{S \in \OS} \left[ \price{S} \right]
        \rightarrow
        \forall \op \left( \op \in \sopt \rightarrow \op \mbox{ is Maximal} \right)
    \end{align}
\end{lemma}

\begin{proof}
    Suppose $\sopt \in \OS$ is a Operation Set with optimal price. Suppose by contradiction that $\exists \op \left( \op \in \sopt \land \op \mbox{ is not Maximal} \right)$, and let $\op = \tuple{b, s}$ be such Operation Pair. By \tref{theorem:max-op-pair}, one of the following cases must be true:
    \begin{enumerate}
        \item $p_s < p_{s+1} \quad \mbox{, if } s+1 < n$
        \item $p_b > p_{b-1} \quad \mbox{, if } b-1 > 0$
    \end{enumerate}

    \subsection*{Case 1} \label{proof:opt-price-I-case-1}
    Suppose that $p_s < p_{s+1}$, $s+1 < n$.
    
    \subsubsection*{Case 1.1}
    Suppose in addition that $\op' = \tuple{b, s+1}$ is independent of all $\op \in \sopt \backslash \Set{\op}$. Let $S' = (\sopt \backslash \Set{\op}) \cup \Set{\op'}$. Notice that 
    \begin{align}
        \price{\op} = \pricep{\op} < \pricep{\op'} = \price{\op'} 
    \end{align}
    Therefore $\price{\sopt} < \price{S'}$ (because $\sopt$ and $S'$ differ only in the elements above), contradicting the optimality of $\sopt$.
    
    \subsubsection*{Case 1.2}
    Suppose in addition that $\exists \op'(\op' \in \sopt \land \b{\op'} = s + 1)$, and let $\op' = \tuple{s+1, s'}$ be such Operation Pair. Let $\op'' = \tuple{b, s'}$ and $S'' = (\sopt \cup {\op''}) \backslash \Set{\op, \op'}$. Notice that
    \begin{align}
        \price{\op} + \price{\op'} & = \\\nonumber
        (\pricep{\op}) + (\pricep{\op'}) & = \\\nonumber
        (p_{s} - p_{b}) + (p_{s'} - p_{s+1}) & = \\\nonumber
        (p_{s'} - p_{b}) + (p_{s} - p_{s+1}) & < \\\nonumber
        p_{s'} - p_{b} & = \\\nonumber
        \price{\op''}
    \end{align}
    Therefore $\price{\sopt} < \price{S''}$ (because $\sopt$ and $S'$ differ only in the elements above), contradicting the optimality of $\sopt$.

    \subsubsection*{Case 1 - Conclusion}
    The hypothesis $p_s < p_{s+1}$, $s+1 < n$ leads to a contradiction. Therefore, if $\sopt$ is optimal, then $p_s \geqslant p_{s+1}$, $s+1 < n$, as we wanted to prove.

    \subsection*{Case 2}
    Suppose that $p_b > p_{b-1}, b-1 > 0$. The proof of this case is similar to the proof of the Case \ref{proof:opt-price-I-case-1}, but this uses $b$ and $b-1$ instead of $s$ and $s-1$.
\end{proof}

\begin{defn}
    Given an $n \in \Nt$, let $\op = \tuple{b, s} \in \Square{n}$ be an Operation Pair. We say that $i$ is included in $\op$, and denote by $\incl{i}{\op}$, when $b \leqslant i \leqslant s$.
\end{defn}

\begin{defn}
    Given an $n \in \Nt$ and a Price Tuple $p$, we say that an Operation Set $S \in \OS$ is \great when:
    \begin{equation}
        \forall i \big(
            \left( i \in \range{n-1} \land p_{i} < p_{i+1}  \right)
            \rightarrow
            \left( \exists \op \left(
                \op \in S \land i \triangleright \op \land (i+1) \triangleright \op
            \right) \right)
        \big)
    \end{equation}
    i.e. the indices of all ascending pairs of $p$ are included in $\op$.
\end{defn}

\begin{lemma}
    \label{lemma:opt-price-II}
    Given an $n \in \Nt$ and a Price Tuple $p$, if a Operation Set $\sopt \in \OS$ is has optimal price, then $\sopt$ is \great.
\end{lemma}

\begin{proof}
    Suppose by contradiction that $\sopt$ is not \great, i.e.
    \begin{equation}
        \exists i \big(
            \left( i \in \range{n-1} \land p_{i} < p_{i+1}  \right)
            \land
            \left( \nexists \op \left(
                \op \in S \land i \triangleright \op \land (i+1) \triangleright \op
            \right) \right)
        \big)
    \end{equation}
    and let $i$ be such value.

    \subsection*{Case 1}
    Suppose that $\incl{i}{\op} \land \neg (\incl{i+1}{\op})$ for some $\op \in \sopt$. If $\incl{i+1}{\op'}$ for some $\op'$, then join $\op$ and $\op'$ to get an Operation Set with cost greater than $\sopt$. If that is not the case, extend $\op$ to include $i+1$ to get an Operation Set with cost greater than $\sopt$. In all cases, the original hypothesis contradicts the optimality of $\sopt$.

    \subsection*{Case 2}
    Suppose that $\neg (\incl{i}{\op}) \land \incl{i+1}{\op}$ for some $\op \in \sopt$. This case is similar to the previous one, except that the Operation Pair has to be extended backwards instead of forwards.

    \subsection*{Case 3}
    Suppose that $\forall \op (\op \in \sopt \rightarrow (\neg (\incl{i}{\op}) \land \neg (\incl{i+1}{\op})))$. Create a new Operation Set $S' = \sopt \Cup \Set{\tuple{i, i+1}}$, which has greater cost, contradicting the optimality of $\sopt$.

    \subsection*{Conclusion}
    The cases above cover all possible cases. Therefore, the lemma has been proven by contradiction.
\end{proof}

\begin{lemma}
    \label{lemma:opt-price-III}
    Given an $n \in \Nt$ and a Price Tuple $p$, if an Operation Set $\sopt \in \OS$ satisfy:
    \begin{enumerate}
        \item $\forall \op \left(
            \op \in \sopt \rightarrow \op \mbox{ is Maximal}
        \right)$
        \item $\sopt$ is \great
    \end{enumerate}
    then $S$ has optimal price, i.e $ \price{S} = \max\limits_{S' \in \OS} [\price{S'}] $.
\end{lemma}

It is tiresome to write this proof so I won't.

\begin{theorem}
    Given an $n \in \Nt$ and a Price Tuple $p$, an Operation Set $\sopt \in \OS$ has optimal price if and only if it satisfies:
    \begin{enumerate}
        \item $\forall \op \left(
            \op \in \sopt \rightarrow \op \mbox{ is Maximal}
        \right)$
        \item $\sopt$ is \great
    \end{enumerate}
\end{theorem}

\begin{proof}
    Lemmas \ref{lemma:opt-price-I} and \ref{lemma:opt-price-II} prove the $\Rightarrow$ part. Lemma \ref{lemma:opt-price-III} proves the $\Leftarrow$.
\end{proof}

\section{Problem Description}

\begin{description}
    \item[Input] a Price Tuple $p$
    \item[Output] a value $z \in \R$
    \item[Goal] $\max z$
\end{description}

% \subsection{Input}

% An array of prices:

% \begin{equation}
%     p = \langle p_{0}, \dots, p_{n-1} \rangle \in \R^n_+
%     \qquad
%     n \in \Nt
% \end{equation}

% \subsection{Intermediate Result}

% An array of ordered pairs, each representing a \textit{buy} ($b$) and $s$ a \textit{sell} ($s$) operation

% \begin{equation}
%     r = \tuple{ \tuple{b_{0}, s_{0}}, \dots, \tuple{b_{m-1}, s_{m-1}}},  m \in \N
% \end{equation}
% satisfying:

% \begin{enumerate}
%     \item $b_{i}, s_{i} \in \{0, \dots, n-1\}, \forall i \in \{0, \dots, m-1\}$
%     \item $b_{i} < s_{i}, \forall i \in \{0, \dots, m-1\}$
%     \item $s_{i} < b_{i+1}, \forall i \in \{0, \dots, m-2\}$
% \end{enumerate}

% \subsection{Output}

% \begin{equation}
%     z = \displaystyle\sum\limits_{i=0}^{m-1} p_{s_{i}} - p_{b_{i}} \in \R_+
% \end{equation}

% \subsection{Goal}

% \begin{equation}
%     \max z
% \end{equation}

\section{Solution - Dynamic Programming}

Given that you buy on a day $i$, while the value does not decrease, you keep it. If it will drop the next day, you sell it.

\subsection{Initial State}

Find the first pair $\langle b, s \rangle$ for which the price increases, i.e. the first pair of consecutive indices for which $p_s - p_b > 0$.

\subsection{Optimal Substructure}

Let:

\begin{enumerate}
    \item $\langle b_l, s_l \rangle$ be the last operation;
    \item $p_{s_l}$ be the price of the last sell;
    \item $i$ the index of the current day;
    \item $p_i$ the stock price of the current day;
    \item $p_{i-1}$ the stock price of the previous day;
\end{enumerate}

Cases:

\begin{enumerate}
    \item if $(s_l == i-1) \land (p_{s_l} \leqslant p_i)$
    \begin{enumerate}
        \item replace $\langle b_l, s_l \rangle$ by $\langle b_l, i \rangle$
        \item rationale: if the stock price is increasing, you keep it;
    \end{enumerate}
    \item if $(s_l < i-1) \land (p_{i-1} \leqslant p_i)$
    \begin{enumerate}
        \item add $\langle i-1, i \rangle$
        \item rationale: if you have no stock and it will increase, you buy and sell it;
    \end{enumerate}
    \item the others are cases in which the stock price drops, and there is nothing to do;
\end{enumerate}

Complexity: $\mathcal{O}(n)$

\section{Solution - Simple}

\begin{algorithm}[H]
    \caption{Simple-Algoruithm}
    \label{algo:ga}
    \begin{algorithmic}[1]
        \State{$ \Delta p \gets \left[ \langle p_{i+1} - p_i \rangle \ for \ i \in \{0, \dots, n-2\}\right]$}
        \State{$\Delta p_> \gets filter(\Delta p, (>=0))$}
        \State{$r \gets sum(\Delta p_>)$}
        \State{\textbf{return} $r$}
    \end{algorithmic}
\end{algorithm}

The filter takes care of removing the drops on the price, while the sum of the differences computes the gains.

Complexity: $\mathcal{O}(n)$
